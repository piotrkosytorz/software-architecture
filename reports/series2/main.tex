%%%%%%%%%%%%%%%%%%%%%%%%%%%%%%
% LATEX-TEMPLATE TECHNICAL REPORT
%-------------------------------------------------------------------------------
% Voor informatie over het technisch rapport, zie
% http://practicumav.nl/onderzoeken/rapport.html
% Voor readme en meest recente versie van het template, zie
% https://gitlab-fnwi.uva.nl/informatica/LaTeX-template.git
%%%%%%%%%%%%%%%%%%%%%%%%%%%%%%

%-------------------------------------------------------------------------------
%	PACKAGES EN DOCUMENT CONFIGURATIE
%-------------------------------------------------------------------------------

\documentclass{uva-inf-article}
\usepackage[utf8]{inputenc}
\usepackage[english]{babel}
\usepackage{dirtytalk}
\usepackage{tabularx}
\usepackage{listliketab}
\usepackage{pdfpages}

% special characters: check-mark and x-mark
\usepackage{amssymb}% http://ctan.org/pkg/amssymb
\usepackage{pifont}% http://ctan.org/pkg/pifont
\newcommand{\cmark}{\ding{51}}%
\newcommand{\xmark}{\ding{55}}%

% Relevant voor refereren vanaf blok 5
\usepackage[
backend=biber,
style=numeric,
citestyle=numeric 
]{biblatex}
\addbibresource{citations.bib} %Imports bibliography file

% Sections structure
\usepackage{titlesec}
\usepackage{hyperref}

\titleclass{\subsubsubsection}{straight}[\subsection]

\newcounter{subsubsubsection}[subsubsection]
\renewcommand\thesubsubsubsection{\thesubsubsection.\arabic{subsubsubsection}}
\renewcommand\theparagraph{\thesubsubsubsection.\arabic{paragraph}} % optional; useful if paragraphs are to be numbered

\titleformat{\subsubsubsection}
  {\normalfont\normalsize\bfseries}{\thesubsubsubsection}{1em}{}
\titlespacing*{\subsubsubsection}
{0pt}{3.25ex plus 1ex minus .2ex}{1.5ex plus .2ex}

\makeatletter
\renewcommand\paragraph{\@startsection{paragraph}{5}{\z@}%
  {3.25ex \@plus1ex \@minus.2ex}%
  {-1em}%
  {\normalfont\normalsize\bfseries}}
\renewcommand\subparagraph{\@startsection{subparagraph}{6}{\parindent}%
  {3.25ex \@plus1ex \@minus .2ex}%
  {-1em}%
  {\normalfont\normalsize\bfseries}}
\def\toclevel@subsubsubsection{4}
\def\toclevel@paragraph{5}
\def\toclevel@paragraph{6}
\def\l@subsubsubsection{\@dottedtocline{4}{7em}{4em}}
\def\l@paragraph{\@dottedtocline{5}{10em}{5em}}
\def\l@subparagraph{\@dottedtocline{6}{14em}{6em}}
\makeatother

\setcounter{secnumdepth}{4}
\setcounter{tocdepth}{4}

%-------------------------------------------------------------------------------
%	GEGEVENS VOOR IN DE TITEL
%-------------------------------------------------------------------------------

% Vul de naam van de opdracht in.
\assignment{Requirements Engineering}
% Vul het soort opdracht in.
\assignmenttype{Report}
% Vul de titel van de eindopdracht in.
\title{Deliverable 1}

% Vul de volledige namen van alle auteurs in.
\authors{Piotr Kosytorz}
% Vul de corresponderende UvAnetID's in.
\uvanetids{UvAnetID 11876964}

% Vul altijd de naam in van diegene die het nakijkt, tutor of docent.
\tutor{drs. Hans Dekker}
% Vul eventueel ook de naam van de docent of vakcoordinator toe.
\docent{}
% Vul hier de naam van de PAV-groep  in.
\group{}
% Vul de naam van de cursus in.
\course{Requirements Engineering}
% Te vinden op onder andere Datanose.
\courseid{}

% Dit is de datum die op het document komt te staan. Standaard is dat vandaag.
\date{\today}


\begin{document}

\section{Series 2}\label{series-2}}

This documents contains our notes and answers to the questions about
software metrics (practical lab Series 2).

Authors

\begin{itemize}
\tightlist
\item
  Cornelius Ries
\item
  Piotr Kosytorz
\end{itemize}

\subsection{About}\label{about}}

TODO

\subsection{Design Desicions}\label{design-desicions}}

TODO

\subsection{Results}\label{results}}

TODO

\subsection{Tool usage}\label{tool-usage}}

To use the tool we provide the source code as a eclipse project

\begin{enumerate}
\def\labelenumi{\arabic{enumi}.}
\tightlist
\item
  Please import the project into your eclipse with a working rascal
  installation.
\item
  Open \texttt{Configuration.rsc} and adjust the location of the
  \texttt{projectLocation} to match the path of the project to your
  eclise
\item
  Do the same for the \texttt{smallSqlProject} and \texttt{hqSqlProject}
\item
  Start a rascal console and import the \texttt{Main} module
\item
  run \texttt{startServe();}
\item
  open a browser and point it towards \texttt{http://localhost:5433} or
  to the location of \texttt{serveAddress} in case you changed it
\end{enumerate}

\subsection{Duplication Detection}\label{duplication-detection}}

The idea and algorithm of our duplication detection is based on the
information from {[}1{]} and {[}2{]}. The main idea behind this approach
is to hash the nodes of an ast into different buckets and collect the
duplications if a bucket has more than 1 element. For type 2 the papers
suggest to clear unneccesary information from the nodes (variable names,
type etc.).

For our implementation we decided to use a map as a utility to do the
matching. We also had to clean the nodes initially because of a change
in rascal that shifted the loc and other informations of a node from
annotations on the node to information contained in the node. This
messed up the matching because every location was unique.

A more detailed explanation can be found in the next chapter.

\subsubsection{How it works
(Pseudocode)}\label{how-it-works-pseudocode}}

\begin{verbatim}
Build the AST of the project.

For all nodes in AST if size > threshold
- Clean nodes for type 1 detection.
- Clean nodes for type 2 detection.
- Collect nodes in map with cleaned node as key, relation of original node and location as value

For all keys in Map build a set of duplications
- Collect all values
- If size of values > 1 add to set

Filter subclones
- For all duplications
- If another duplication exists for which all locations include the locations of the current one Then
    -
  Else
    Add to new Set
    
For all filtered clones
- Collect them in output format
\end{verbatim}

\subsection{Visualization}\label{visualization}}

TODO

\subsection{Tests}\label{tests}}

All tests are in seperate files that extend their original rascal
module:

\begin{itemize}
\tightlist
\item
  DuplicationsAnalyzerTests
\item
  RaterTests
\item
  UtilsTests
\item
  VolumeAnalyzerTests
\end{itemize}

To run the tests, import all the modules above and execute :test in the
rascal console. The \texttt{projectLocation} in
\texttt{Configuration.rsc} has to be set to the projects location in
your eclipse!

\subsection{References}\label{references}}

{[}1{]} Lazar, F.-M. \& Banias, O., 2014. Clone detection algorithm
based on the Abstract Syntax Tree approach. Applied Computational
Intelligence and Informatics (SACI), 2014 IEEE 9th International
Symposium on, pp.73--78.

{[}2{]} Baxter et al., 1998. Clone detection using abstract syntax
trees. Software Maintenance, 1998. Proceedings., International
Conference on, pp.368--377.

\end{document}