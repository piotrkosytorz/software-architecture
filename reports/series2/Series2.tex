%%%%%%%%%%%%%%%%%%%%%%%%%%%%%%
% LATEX-TEMPLATE TECHNISCH RAPPORT
%-------------------------------------------------------------------------------
% Voor informatie over het technisch rapport, zie
% http://practicumav.nl/onderzoeken/rapport.html
% Voor readme en meest recente versie van het template, zie
% https://gitlab-fnwi.uva.nl/informatica/LaTeX-template.git
%%%%%%%%%%%%%%%%%%%%%%%%%%%%%%

%-------------------------------------------------------------------------------
%	PACKAGES EN DOCUMENT CONFIGURATIE
%-------------------------------------------------------------------------------

\documentclass{uva-inf-article}
\usepackage[english]{babel}

\usepackage[
backend=biber,
style=numeric,
citestyle=numeric 
]{biblatex}

\usepackage{xcolor}
\usepackage{geometry}
\newcommand\todo[1]{\textcolor{red}{#1}}

\addbibresource{citations.bib}

%-------------------------------------------------------------------------------
%	GEGEVENS VOOR IN DE TITEL
%-------------------------------------------------------------------------------

% Vul de naam van de opdracht in.
\assignment{Software Evolution Series 2}
% Vul het soort opdracht in.
\assignmenttype{Report}
% Vul de titel van de eindopdracht in.
\title{Report}

% Vul de volledige namen van alle auteurs in.
\authors{Cornelius Ries; Piotr Kosytorz}
% Vul de corresponderende UvAnetID's in.
\uvanetids{11884827; 11876964}

% Vul altijd de naam in van diegene die het nakijkt, tutor of docent.
\tutor{Riemer van Rozen}
% Vul eventueel ook de naam van de docent of vakcoordinator toe.
\docent{}
% Vul hier de naam van de PAV-groep  in.
% \group{Group 7}
% Vul de naam van de cursus in.
\course{Software Evolution}
% Te vinden op onder andere Datanose.
\courseid{}

% Dit is de datum die op het document komt te staan. Standaard is dat vandaag.
\date{\today}

%-------------------------------------------------------------------------------
%	VOORPAGINA
%-------------------------------------------------------------------------------

\begin{document}
\maketitle

%-------------------------------------------------------------------------------
%	INHOUDSOPGAVE EN ABSTRACT
%-------------------------------------------------------------------------------

% Niet doen bij korte verslagen en rapporten
%\tableofcontents
%\begin{abstract}
%\lipsum[13]
%\end{abstract}

%-------------------------------------------------------------------------------
%	INTRODUCTIE
%-------------------------------------------------------------------------------

\section{Series 2}

This documents contains our notes and answers to the questions about
software metrics (practical lab Series 2).

\subsection{About}

\todo{TODO: Write the "about" section}

\subsection{Design Decisions}

\todo{TODO}

\subsection{Results}

\todo{TODO}

\subsection{Tool usage}

To use the tool we provide the source code as a eclipse project.

\begin{enumerate}

\item
  Please import the project into your eclipse with a working rascal
  installation.
\item
  Open \texttt{Configuration.rsc} and adjust the location of the
  \texttt{projectLocation} to match the path of the project to your
  eclise
\item
  Do the same for the \texttt{smallSqlProject} and \texttt{hqSqlProject}
\item
  Start a rascal console and import the \texttt{Main} module
\item
  run \texttt{startServe();}
\item
  open a browser and point it towards \texttt{http://localhost:5433} or
  to the location of \texttt{serveAddress} in case you changed it
\end{enumerate}

\subsection{Duplication Detection}

The idea and algorithm of our duplication detection is based on the
information from \cite{lazar2014clone} and \cite{baxter1998clone} The main idea behind this approach
is to hash the nodes of an AST into different buckets and collect the
duplications if a bucket has more than 1 element. For type 2 the papers
suggest to clear unnecessary information from the nodes (variable names,
type etc.).

For our implementation we decided to use a map as a utility to do the
matching. We also had to clean the nodes initially because of a change
in rascal that shifted the loc and other informations of a node from
annotations on the node to information contained in the node. This
messed up the matching because every location was unique.

A more detailed explanation can be found in the next chapter.

\subsubsection{How it works (Pseudocode)}

\begin{verbatim}
Build the AST of the project.

For all nodes in AST if size > threshold
- Clean nodes for type 1 detection.
- Clean nodes for type 2 detection.
- Collect nodes in map with cleaned node as key, relation of original node and location as value

For all keys in Map build a set of duplications
- Collect all values
- If size of values > 1 add to set

Filter subclones
- For all duplications
- If another duplication exists for which all locations include the locations of the current one Then
    -
  Else
    Add to new Set
    
For all filtered clones
- Collect them in output format
\end{verbatim}

\subsection{Visualization}

\todo{TODO}

\subsection{Tests}

All tests are in seperate files that extend their original rascal
module:

\begin{itemize}
\item
  DuplicationsAnalyzerTests
\item
  RaterTests
\item
  UtilsTests
\item
  VolumeAnalyzerTests
\end{itemize}

To run the tests, import all the modules above and execute :test in the
rascal console. The \texttt{projectLocation} in
\texttt{Configuration.rsc} has to be set to the projects location in
your eclipse!


%-------------------------------------------------------------------------------
%	REFERENTIES
%-------------------------------------------------------------------------------

\printbibliography

%-------------------------------------------------------------------------------
%	BIJLAGEN EN EINDE
%-------------------------------------------------------------------------------

%\section{Bijlage A}
%\section{Bijlage B}
%\section{Bijlage C}
\end{document}
